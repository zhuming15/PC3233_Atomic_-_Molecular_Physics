\documentclass{article}

\usepackage[a4paper, total={6in, 8in}]{geometry}
\usepackage[breakable]{tcolorbox}
\usepackage{parskip} % Stop auto-indenting (to mimic markdown behaviour)
\usepackage{graphicx} % Required for inserting images
\usepackage{physics}
\usepackage{amsthm}
\usepackage{amssymb}
\usepackage{mathtools}
\usepackage{amsmath}

\title{PC3233 Assignment 2}
\author{Hor Zhu Ming (A0236535A)}
\date{Feb 2024}

\begin{document}

\maketitle

\section{The Radial Equation}
Consider the following potential and differentiation:
\begin{equation}
  \dv[2]{u}{r} + 2(\frac{1}{r} - \kappa) \dv{u}{r} + \left( \frac{2a-2\kappa}{r} - \frac{l(l+1)}{r^2}\right) u = 0
\end{equation}

Ansatz: Assume the general solution of $u(r) = \sum_{j} b_j r^{j}$
Then, we have
\begin{equation}
  \begin{cases}
    \displaystyle
    u(r) = \sum_{j = 0} b_j r^{j} \\
    \displaystyle
    \dv{u}{r} = \sum_{j = 0} b_j j r^{j-1} \\
    \displaystyle
    \dv[2]{u}{r} = \sum_{j = 0} b_j j (j-1) r^{j-2} \\
  \end{cases}
\end{equation}

Since any $r^{j}$ term with $j < 0$ is zero, we can rewrite the differentiation as

\begin{equation}
  \begin{cases}
    \displaystyle
    u(r) = \sum_{j = 0} b_j r^{j} \\
    \displaystyle
    \dv{u}{r} = \sum_{j = 1} b_j j r^{j-1} = \sum_{j = 0} b_{j+1} (j+1) r^{j} \\
    \displaystyle
    \dv[2]{u}{r} = \sum_{j = 2} b_j j (j-1) r^{j-2} = \sum_{j = 0} b_{j+2} (j+2) (j+1) r^{j} \\
  \end{cases}
\end{equation}

Substituting the result $(3)$ into the original equation $(1)$, we have

\begin{equation*}
    \sum_{j = 0} b_{j+2} (j+2) (j+1) r^{j} + 2(\frac{1}{r} - \kappa) \sum_{j = 0} b_{j+1} (j+1) r^{j} + \left( \frac{2a-2\kappa}{r} - \frac{l(l+1)}{r^2}\right) \sum_{j = 0} b_j r^{j} = 0
\end{equation*}

Then expanding the summation, we have

\begin{equation*}
  \sum_{j = 0} b_{j+2} (j+2) (j+1) r^{j} + 2\sum_{j = 0} b_{j+1} (j+1) r^{j-1} - 2\kappa\sum_{j = 0} b_{j+1} (j+1) r^{j} + (2a-2\kappa) \sum_{j = 0} b_j r^{j-1} - l(l+1)\sum_{j = 0} b_j r^{j-2}  = 0 \\
\end{equation*}

Similarly, we can shift the index of the summation to get for any $r^{j}$ with $j < 0$, we have

\begin{equation*}
  \sum_{j = 0} b_{j+2} (j+2) (j+1) r^{j} + 2\sum_{j = 1} b_{j+1} (j+1) r^{j-1} - 2\kappa\sum_{j = 0} b_{j+1} (j+1) r^{j} + (2a-2\kappa) \sum_{j = 1} b_j r^{j-1} - l(l+1)\sum_{j = 2} b_j r^{j-2}  = 0 \\
\end{equation*}

Then, we can rewrite the summation as

\begin{equation*}
  \sum_{j = 0} b_{j+2} (j+2) (j+1) r^{j} + 2\sum_{j = 0} b_{j+2} (j+2) r^{j} - 2\kappa\sum_{j = 0} b_{j+1} (j+1) r^{j}+ (2a-2\kappa) \sum_{j = 0} b_{j+1} r^{j} - l(l+1)\sum_{j = 0} b_{j+2} r^{j}  = 0 \\
\end{equation*}

Then, we can combine the summation and factor out the $r^{j}$ term to get

\begin{equation}
  \sum_{j = 0} \left[ b_{j+2} (j+2) (j+1) + 2b_{j+2} (j+2) - 2\kappa b_{j+1} (j+1) + (2a-2\kappa) b_{j+1} - l(l+1) b_{j+2} \right] r^{j}  = 0 \\  
\end{equation}

In order for the equation to hold for all $r^{j}$, we must have

\begin{equation}
  b_{j+2} (j+2) (j+1) + 2b_{j+2} (j+2) - 2\kappa b_{j+1} (j+1) + (2a-2\kappa) b_{j+1} - l(l+1) b_{j+2} = 0
\end{equation}  

For all $j$. Then, we can rewrite the equation as

\begin{equation}
  b_{j+2} \left[ (j+2) (j+1) + 2(j+2) - l(l+1) \right] = - 2\kappa b_{j+1} (j+1) + (2a-2\kappa) b_{j+1}
\end{equation}

Then, we can factor out the $b_{j+2}$ and $b_{j+1}$ term to get

\begin{equation}
  b_{j+2} \left[ (j+2) (j+1) + 2(j+2) - l(l+1) \right] = b_{j+1} [(2a-2\kappa) - 2\kappa (j+1)]
\end{equation}

Then, we can simplify the equation to get

\begin{align}
  b_{j+2} &= 2 b_{j+1} \frac{(a-\kappa) - \kappa (j+1)}{(j+2) (j+1) + 2(j+2) - l(l+1)} \\
  &= 2 b_{j+1} \frac{k(j+2) - a}{(j+2) (j+3) - l(l+1)} \\
\end{align}

Then, rewrite the index $j+2$ as $j$ to get

\begin{equation}
  b_{j} = 2 b_{j-1} \frac{\kappa j - a}{(j) (j+1) - l(l+1)}
\end{equation}

In order for the series to not diverge, Then

\begin{equation}
  j \geq l + 1 \implies n \geq l + 1
\end{equation}



\section{The Hydrogen Atom and Spin}
\subsection{Part A}
"eV" stands for electron-volt, which is a unit of energy. It is defined as the energy gain for an single electron
 when it is accelerated through an electric potential difference of $1$ volt.

\begin{equation}
  \text{Work done (In Joule), } W = qV \text{ (In eV)} 
\end{equation}
Then,
\begin{equation}
  1 \text{eV} = 1.602 \times 10^{-19} \text{J, Joule}
\end{equation}

\subsection{Part B}
\begin{equation}
  \mu = \frac{m_em_p}{m_e + m_p} = \frac{9.10938 \times 10^{-31} \cdot 1.67262 \times 10^{-27}}{9.10938 \times 10^{-31} + 1.67262 \times 10^{-27}} = 9.103 \times 10^{-31}
\end{equation}

Since the energy levels are quantized, the energy levels are given by

\begin{equation}
  E_n = -\frac{m_e e^4}{8(h \epsilon_0)^2} \frac{1}{n^2} = - R^{*}_y \frac{Z^2}{n^2}
\end{equation}

Where $R^{*}_y = \frac{m_e e^4}{8(h \epsilon_0)^2}$ is the Rydberg constant for hydrogen-like atoms which is $13.6$ eV.

For Hydrogen atom, the $Z$ is $1$, since it has only one proton. Then, the energy levels are given by

\begin{equation}
  \begin{cases}
    \displaystyle
    E_1 = -13.6 \frac{1}{1^{2}} = -13.6 \text{eV} \\
    \\
    \displaystyle
    E_2 = -13.6 \frac{1}{2^{2}} = -3.4 \text{eV} \\
    \\
    \displaystyle
    E_3 = -13.6 \frac{1}{3^{2}} = -1.51 \text{eV} \\
  \end{cases}
\end{equation}

To calculate the electric current and the magnetic dipole moment, we can use the following equations

\begin{equation}
  \begin{cases}
  \displaystyle
  \mu_{l} = I \cdot \vec{A} = I \cdot \pi r^2 \cdot \vec{n} \\
  \\
  \displaystyle
  \mu_{l} = -\frac{u_B}{\hbar} \cdot \vec{L}
  \end{cases}
\end{equation}

Then we can rearrange the equation to get

\begin{equation}
  I = - \frac{u_B}{\pi r^{2}} \abs{\vec{L}}
\end{equation}

The magnitude of the angular momentum is given by

\begin{equation}
  \abs{\vec{L}} = \sqrt{l(l+1)} \hbar
\end{equation}

The maximum value of the magnetic dipole moment is given by the maximum value of $l$ which can determine
by the quantum number $n$. Then, the maximum value of $l$ is $n-1$. 

Since we know that $r_n$ is given by

\begin{equation}
  r_n = \frac{4\pi \epsilon_{o} \hbar^2}{m_e e^2} n^{2}
\end{equation}

\begin{equation}
  \begin{cases}
    r_1 = 5.29 \times 10^{-11} \text{m} \\
    r_2 = 4 \times 5.29 \times 10^{-11} \text{m} = 2.12 \times 10^{-10} \text{m} \\
    r_3 = 9 \times 5.29 \times 10^{-11} \text{m} = 4.76 \times 10^{-10} \text{m} \\
  \end{cases}
\end{equation}

Then we can find

\begin{equation}
  \begin{cases}
    \displaystyle
    \mu_{n = 1, l = 0} = 0 \implies I_{n =1, l = 0} = 0 \\
    \\
    \displaystyle
    \mu_{n = 2, l = 1} = \sqrt{2} \mu_B \implies I_{n = 2, l = 1} = \frac{\sqrt{2}\mu_B}{2 \pi r_{2}^{2}} \\
    \\
    \displaystyle
    \mu_{n = 3, l = 2} = \sqrt{6} \mu_B \implies I_{n = 3, l = 2} = \frac{\sqrt{6}\mu_B}{2 \pi r_{3}^{2}} \\
  \end{cases}
\end{equation}

\subsection{Part C}

The magnetic moment of electron is given by

\begin{equation}
  \mu_e = - 9.284 \times 10^{-24} \text{J/T}
\end{equation}

Since $\mu_e = \frac{e}{2 m_e}L = - \frac{erv}{2}$, we can solve for $v$ to get by assuming $r = 10^{-18}$m

\begin{align}
  v &= - \frac{2 \mu_e}{er} \\
  &= - \frac{2 \cdot -9.284 \times 10^{-24}}{1.6 \times 10^{-19} \cdot 10^{-18}} \\
  &=  1.159 \times 10^{14} \text{m/s} \\
\end{align}

This is a very high speed, which is exceeded to the speed of light. This is not possible since the speed of light is the maximum speed in the universe.
\end{document}
